\section{Jaká práva má autor počítačového programu a~jak jsou tato autorská práva omezena? Jaké jsou varianty výkonu v těchto práv v závislosti na způsobu vzniku počítačového programu (školní dílo, zaměstnanecké dílo, spoluautorské dílo)?}

Počítačový program není přesně definován ale popisuje se jako \uv{\emph{program v jákekoliv formě, včetně těch, které jsou součástí technického vybavení (HW)}}. Jak na SK tak EU patentový úřad jej vyjadřuje jako \uv{\emph{serii instrukcí, kterou lze spustit na PC}}.

Autor má autorská práva, kde jsou tyto práva chráněna jako literální dílo a to bez ohledu na formu vyjádření.
\newline

\noindent Ochrana pokrývá vyjádření ve formě:
\begin{itemize}[noitemsep]
    \item strojový kód
    \item zdrojový kód
    \item jejich mezistupně
    \item přípravné koncepční materiály vzniklé při vývoji
    \begin{itemize}[noitemsep]
        \item model architektůry SW
        \item funkční specifikace
        \item apod.
    \end{itemize}
\end{itemize}

Vyloučené z ochrany jsou \uv{\emph{myšlenky a principy na nichž je založen jakýkoliv prvek PC programu, včetně těch, které jsou podkladem jeho propojení s jiným programem}}. Neboli není chráněna funkcionalita programu, ale pouze její objektivní vyjádření v podobě příslušného kódu.

\textbf{Školní dílo} (\href{https://www.zakonyprolidi.cz/cs/2000-121#p60}{Autorský zákon §60}) -- škola má za obvyklých podmínek právo na uzavření licenční smlouvy o užití díla. Pokud není závažný důvod, měl by autor udělit licenci, nebo bude udělena soudem.

\textbf{Zaměstnanecké dílo} (\href{https://www.zakonyprolidi.cz/cs/2000-121#p58}{Autorský zákon §58}) -- jestliže zaměstnanec vytvoří program ke splnění svých povinností vyplývajících z pracovněprávního nebo služebního vztahu a neexistuje-li mezi zaměstnavatelem a zaměstnancem odlišná dohoda, zaměstnavatel vykonává k takovému programu svým jménem a na svůj účet autorova majetková práva, kde je autor kompenzován nejčastěji mzdou. Pokud není dohoda odlišná, zaměstnanec současně uděluje zaměstnavateli svolení k úkonům, které by zasahovaly do osobních práv.
\begin{itemize}[noitemsep]
    \item spojování s jinýmy programy
    \item provádění dodatečných změn
    \item uvedení na trh
    \item vše výše bez zvláštního souhlasu zaměstnance
\end{itemize}

\textbf{Spoluautorské dílo} (\href{https://www.zakonyprolidi.cz/cs/2000-121#p8}{Autorský zákon §8}) je dílo na kterém se podílí více autorů, kdy všem zúčastněným autorům připadá stejné právo. O nakládaní s dílem musí být rozhodnuto jednomyslně. O spoluautorské dílo se jedná pokud jednotlivé části nejsou způsobilé samostatného užití, jedna část potřebuje druhou. Z právních úkonů jsou také vázáni společně a nerozdílně.

\textbf{Kolektivní dílo} (\href{https://www.zakonyprolidi.cz/cs/2000-121#p59}{Autorský zákon §59}) je dílo, na kterém se podílí více autorů a je vytvořeno z podnětu a pod vedením FO/PO a uváděno na veřejnost pod jejím jménem. Považují se často za zaměstnanecká díla.


\section{Jaké jsou zákonné a judikatorní podmínky pro dovolené reverzní inženýrství software?}

Dekompilace programu, není povolena v každém případě. Lze provádět pouze za účelem interoperability (je schopnost různých systémů vzájemně spolupracovat, poskytovat si služby, dosáhnout vzájemné součinnosti) a je třeba provádět jen nezbytně nutné úkony. Dekompilovat lze pouze až po vyčerpání všech možností jako je obrácení se na autora. Platí pouze za speciálních podmínek. Při dekompilaci lze pouze provádět rozmnožovaní kódu a překlad formy kódu ve smyslu čl. 4 SW směrnice -- stále nebo dočasné rozmnožovaní, překlady zpracovaní, a jiné úpravy programu. Taktéž nelze dekompilovat pokud program, se kterým chceme dosáhnout interoperability neexistuje alespoň ve formě návrhu. Dále může dekompilovat pouze oprávněná osoba, nelegální držitelé licence dekompilaci nemohou provádět. Všechny potřebné informace mohou být požity pouze k dosažení interoperability. 

\textbf{Oprávněný nabyvatel} -- může být ten, kdo si program zakoupil, pronajal, získal licenci přímo od držitele práv, a i od původního oprávněného nabyvatele.

\textbf{Oprávněný uživatel} -- problematické, oprávněný uživatel může být na základě smlouvy, nebo zákona §66 \uv{\emph{Oprávněným uživatelem je oprávněný nabyvatel rozmnoženiny, který k ní má vlastnické nebo jiné právo za účelem využití}}. Řešil to nejvyšší soud i soudní dvůr EU, který stanovil, že se může jednat i o jinou osobu než o tu se kterou byla smlouva uzavřena (lze přeprodat licenci), ale původní uživatel musí odstranit SW ze svého PC a nepoužívat jej. Nejvyšší soud se zabýval poslaným CD s licenčním klíčem.

\textbf{Třístupňový test} -- hledání výjimky, platné jsou pouze výjimky stanovené v zákoně, aplikace vyjímek je dovolena jen, pokud to není v rozporu s užitím díla a nejsou tím nepřiměřeně dotčeny oprávněné zájmy autora. I při dovoleném reverzním inženýrství musíme projít tímto testem.

\textbf{Blackbox analýza} -- mohu zkoumat nebo studovat funkčnost programu za účelem zjištění myšlenek a principů na nichž je založen jakýkoliv prvek programu. (\href{https://www.zakonyprolidi.cz/cs/2000-121#p66}{Autorský zákon §66})


\section{Jak mohou být právem chráněna rozhraní (datová, uživatelská, aplikační)? Je právem chráněna funkcionalita software?}

Je uvedeno, že myšlenky a principy, na nichž je založen prvek programu, nejsou chráněny autorským zákonem. SW směrnice -- Myšlenky a zásady na kterých je založen kterýkoliv prvek programu, nebo jeho rozhraní, nejsou chráněny. 

\textbf{Datová} -- jedná se o rozhraní, která slouží k ukládaní a přenosu dat v určitém formátu. Dle SW směrnice tyto rozhraní nebudou chráněna. Pokud by bylo bráno jako normální dílo, lze uvažovat o standardní ochraně.

\textbf{Uživatelská (GUI)} -- Při převzetí GUI se nejedná o rozmnoženinu počítačového programu. \textrightarrow Nejedná se o zásah do vyhrazeného práva. Pokud bychom ale naplnili při tvorbě GUI podmínku jedinečnosti, může být chráněno podle obecné autorskoprávní ochrany. Vzniká problém u právních systémů neoperujících s pojmem originálnost, např. UK\@. Tam GUI chráněna nejsou.

\textbf{Aplikační (API)} -- Není specifikováno, zda se jedná o myšlenku nebo vyjádření. SDEU neřešil ale lze předpokládat, že by to bylo podobné jako u GUI\@. Pokud by se jednalo o vyjádření tak je chráněno. Momentálně Google vs Oracle, o kterém by měl soud rozhodnout v 2021. Google pro Android využil stejné API jako je v Javě. V prvním řešení bylo rozhodnuto že API není chráněno. Poté odvolací soud rozhodl že API je chráněno, že byla splněna podmínka originality. Momentálně je případ u nejvysšího soudu USA\@.


\section{Jaké jsou základní rozdíly mezi autorskoprávní a patentovou ochranou? Lze chránit v ČR software jako takový? Jaké jsou podmínky patentovatelnosti vynálezu realizovaného počítačem (computer implemented invention)?}

\textbf{Autorskoprávní ochrana} programů je neschopna chránit funkcionalitu daného programu. Dokáže chránit pouze objektivní vyjádření v kódu, popřípadě jeho vizuální stránku, ale samostatná funkcionalita není chráněna. K přiměřené ochraně samostatné funkcionality je použita \textbf{patentová ochrana}. Patentově nelze chránit počítačové programy ale tzv.\ vynálezy uskutečňované počítačem.
\newline

\noindent Podmínky patentovatelnosti:
\begin{itemize}[noitemsep]
    \item vynález (ne program!) realizovaný počítačem
    \item novost -- vynález se považuje za nový, není-li součástí stavu techniky
    \item výsledek vynálezecké činnosti
    \item průmyslová využitelnost -- může-li být vynález vyráběn nebo využíván ve všech odvětvých průmyslu
\end{itemize}
Obecně v ČR SW je chráněn pouze autorským zákonem, pokud nespadá pod vynálezy uskutečňované počítačem, ty lze chránit patenty.  
Vynálezy uskutečňované počítačem jsou chráněny i v ČR. Dle zákona nelze patentovat \uv{\emph{plány, pravidla a způsob vykonávaní duševní činnosti, hraní her nebo vykonávání obchodní činnosti, jakož i programy počítačů}}.

Do vynálezu realizovaného počítačem lze zahrnout počítačový program myšlený jako produkt. Podmínkou je technický charakter příslušného vynálezu. Počítačový program je vynález realizovaný počítačem, jestliže je schopný vyvolat dodatečný technický účinek, když běží na PC nebo je na něm nahrán. Nesmí se jednat o běžnou interakci mezi SW a HW\@. Nelze patentovat SW, jako takový ale musí splňovat určité podmínky, jejich společným jmenovatelem je přítomnost dalšího technického prvku. 


\section{Jak lze chránit v ČR databáze? Jak Soudní dvůr Evropské unie vykládá pojem \uv{\emph{podstatný vklad do pořízení, ověření nebo předvedení obsahu databáze}} ve vztahu k přiznání ochrany zvláštním právem pořizovatele databáze?}

Databázi lze definovat jako \uv{\emph{vnitřně organizované soubory informací, údajů, dat, tedy soubory poznatků o jakýchkoliv skutečnostech}}.

Databáze podle \href{https://www.zakonyprolidi.cz/cs/2000-121#p88}{AZ §88} -- \uv{\emph{soubor nezávislých děl, údajů, nebo jiných prvků systematicky nebo metodicky uspořádaných a individuálně přístupných elektronickými nebo jinými prostředky, bez ohledu na formu jejich vyjádření}}.

V ČR lze databáze chránit dvěma způsoby: pomocí autorského zákona a ochranu právem sui generis -- zvláštní právo zřizovatele databáze.

Z pohledu autorského zákona nejprve databáze ovlivnila Bernská úmluva, a poté TRIPS (specifikuje, že chrání způsobem výběru či uspořádáním originální soubory libovolných materiálů a dat) na kterou navázala WCT, která řeší vše až na problematiku dočasné technické rozmnoženiny. Autorským zákonem je chráněna struktura databáze a ne její obsah. Autorem databáze může být pouze FO nebo skupina takových osob, ale také pokud to umožňuje právní řád tak PO\@.

Ochrana zvláštním právem zřizovatele databáze. Zřizovatelem databáze je osoba, která na svou odpovědnost databázi pořídí, nebo osoba, pro kterou z jejího podnětu učiní jiná osoba. Zřizovatel databáze má právo na vytěžování nebo zužitkovaní celého obsahu databáze, nebo její kvalitativné či kvantitativní části a udělit toto oprávnění i jiné osobě. 

Vytěžováním se rozumí trvalé přenesení jejího obsahu na jiný nosič. Na elektro databáze se nevztahuje výjimka pro osobní použití. Co se týče vytěžení jen části databáze, záleží, jestli se jedná o podstatnou, nebo nepodstatnou část, jelikož nepodstatnou část lze vytěžit a zužitkovat dle směrnice i autorského zákona.

SDEU si vykládá, že při posuzovaní, zda se jedná o vytěžovaní podstatné či nepodstatné části, není aktuální hodnota údajů relevantní. Rozhodné je jaké finanční prostředky či úsilí byly vloženo do pořízení, ověření nebo převedení obsahu této části a nikoliv jaká je cenová hodnota těchto dat v době jejich vytěžení či zužitkování. Také vyjádřil že to, co nepředstavuje podstatnou část, představuje nepodstatnou. Platí to do chvíle, než je databáze vytěžována systematicky.


\section{Může být počítačový program autorem počítačového programu? Jak se aktuálně právo staví k výtvorům umělé inteligence?}

Momentálně dle autorského zákona může být autorem pouze FO -- pokud SW vytvoří SW tak primární SW není autorem sekundárního SW. Pokud je při vývoji SW vysoká účast člověka, je sekundární SW brán jako výsledek lidské činnosti a AI je použito jako pomocník při vývoji. V tom případě je sekundární SW chráněn autorským právem. Při nízké účasti člověka není rozhodnutí jednoznačné. Porovnávají se prvky objektivní vnímatelnosti, tvůrčí svobody, možnosti vyjádření apod. Obecně AI jejich výtvorů není definována. 

Rozlišujeme dva druhy AI -- silné a slabé. Slabá AI -- nefunguje jako blackbox, neboli dokážeme predikovat výstup a víme, jak funguje na pozadí. Silná AI -- funguje jako blackbox takže nevíme co se děje na pozadí. U slabé AI je vlastníkem licence autor AI, zatímco při silné AI se žádný případ doposud neřešil, ani není stanovené nějaké právo takže se momentálně nedá jednoznačně říci.

\textbf{SW paradox} -- máme vývojáře, který vytvoří AI1 a to licencuje uživateli 1. AI1 vytvoří AI2 a tu bude využívat uživatel 2 a v té chvíli vzniká otázka, kdo ho má licencovat. Jestli vývojáři, uživatel 1 nebo AI1. Neví se přímo, na koho se obrátit a jestli nevyžadovat i licenci k AI1. Z tohoto plyne jakási nejistota o platnosti licence. Autorské právo je postaveno na objektivním pravidle. Kdo něco vytvořil tvůrčí činností, je autor a může produkt licencovat. Jelikož si nejsme jistí od koho licencovat a jestli se vůbec jedná o tvůrčí výsledek, může tato skutečnost paralyzovat jakoukoliv licenční smlouvu a její užívaní.

AI jako objekt práva nemá právní osobnost a způsobilost. Momentálně chybí přesvědčivý argument. 

Shrnutí: nejvíce relavantními jsou nároky tvůrců AI a uživatelů -- něco jako spoluautorství. AI může být \uv{tvůrčí}, ale ne autor.

\newpage
\section{Co je to licenční smlouva? Vyjmenujte podstatné a pravidelné náležitosti a popište účel uzavření licenční smlouvy ve vztahu k distribuci software.}

Licenční smlouva je smlouva, na jejíž základě poskytovatel poskytuje oprávnění k užítí všech nebo jednotlivých způsobů užití. Nabyvatel se zavazuje poskytnout odměnu, není-li sjednáno jinak. 

Licenční smlouva nemusí být v písemné formě. Lze ji uzavřít například ústně. Musí být uzavřena písemně pouze v případech, kdy je poskytována jako výhradní. V případě výhradní licence autor nesmí poskytnout licenci třetí osobě a je obvykle povinen nepoužívat SW, ke kterému výhradní licenci udělil. V případě nevýhradní licence může autor používat SW a k obsahu licence poskytnout licence dalším osobám. 
\newline

\noindent\textbf{Elektronické uzavíraní:}
\begin{itemize}[noitemsep]
    \item Click-wrap
    \begin{itemize}[noitemsep]
        \item Potvrzení před prvním užitím
    \end{itemize}
    \item Shrink-wrap
    \begin{itemize}[noitemsep]
        \item Rozbalení krabicového SW
    \end{itemize}
    \item Browse-wrap
    \begin{itemize}[noitemsep]
        \item Souhlas před stažením SW
    \end{itemize}
\end{itemize}

\noindent\textbf{Obsah licence:}
\begin{itemize}[noitemsep]
    \item Základní
    \begin{itemize}[noitemsep]
        \item Smluvní strany
        \item Specifikace autorského díla
        \begin{itemize}[noitemsep]
            \item Předmět
            \item Není nutno popisovat funkcionalitu
        \end{itemize}
        \item Právo a způsob užití
        \item Rozsah licence
        \item Odměna za poskytnutí licence
        \item Přiměřená dodatečná odměna
        \item Délka trvání licence
    \end{itemize}
    \item Ostatní
    \begin{itemize}[noitemsep]
        \item Právo podlicencování, či přeprodání
        \item Odpovědnost za škodu a právní vady SW
        \item Oprávnění k rozmnožovaní nebo úpravě SW
        \item Nárok na upgrade SW
        \item Způsob zániku licence a postupu po zániku
        \item Automatické prodlužovaní licence
    \end{itemize}
\end{itemize}

Hlavním účelem je ochrana díla a specifikace, jak s ním lze nakládat. Například jestli lze upravovat nebo předělávat SW, rozmnožovat ho a popřípadě upravenou verzi licencovat. 


\section{Co je to smlouva o dílo? Vyjmenujte podstatné a pravidelné náležitosti a popište účel uzavření licenční smlouvy ve vztahu k vývoji software.}

\textbf{Obsah smlouvy o dílo:}
\begin{itemize}[noitemsep]
    \item Zhotovitel a zadavatel/objednatel
    \item Předmět smlouvy -- co zadavatel chce
    \item Cena
    \item Termín zhotovení -- do kdy a co bude předáno
    \item Detailnější specifikace předání a převzetí díla
    \item Odpovědnost za vady
    \item Závěrečná ustanovení
\end{itemize}

Je to smlouva, na jejímž základě vzniká závazek, jehož předmětem je zhotovení, údržba, oprava nebo úprava věci nebo činnosti. Většinou se sepisuje pokud je činnost financovaná zákazníkem a produkt (v našem případě SW) je vytvářen dle požadavků zákazníka. Často se pojí se SW na zakázku.

Klíčovým ujednáním je specifikace předmětu plnění neboli SW\@. Oproti specifikaci standardního SW musí být míra specifikace o výrazně rozsáhlejší a detailnější. Přesná specifikace musí proběhnout, aby se předešlo sporům. Proto by specifikace neměla pokrývat pouze funkcionalitu ale také parametry, které nejsou přímo spojené s funkcionalitou ale mohou ji zásadně ovlivnit, jako například HW nároky.

Účelem uzavření licence by měla být nejen schopnost ho používat ale i různě s ním manipulovat. Například nemožnost přeprodávat nebo poskytovat SW třetí straně. Zde se aplikuje občanský zákoník, kde má zákazník možnost používat licenci pouze za účelem sjednaného ve smlouvě. Proto se často sjednává širší oprávnění zákazníka/široká licence. 

Uzavření licenční smlouvy je za účelem využití daného SW, kde je specifikováno co vše lze se SW dělat. Například přeprodej třetím stranám, neomezené množství vlastních licencí na neomezenou dobu, zákaz dodavateli aby program sám užíval nebo prodával licenci jiným osobám, popřípadě právo měnit a upravovat dodaný SW. Dále lze sjednat přístup ke zdrojovým kódům + ošetřit její kvalitu (komentáře, dokumentace, atd.).


\newpage
\section{Co je to SLA\@? Vyjmenujte náležitosti a popište účel uzavření SLA.}

SLA (= Service Level Agreement) je tzv.\ inominátní smlouva, která upravuje úroveň poskytovaní určité služby. Předmětem můžou být služby jako podpora, údržba a podobné spojené s dodávkou SW, služby v oblasti cloud computingu nebo služby v oblasti telekomunikací. Často není uzavíraná samostatně, ale je spíše doplňující smlouva.
\newline

\noindent\textbf{Typické prvky:}
\begin{itemize}[noitemsep]
    \item Vymezení samostatné služby, tedy její definice
    \begin{itemize}[noitemsep]
        \item Podpora SW
        \item Odstraňovaní vad
        \item PC program poskytovaný jako služba
    \end{itemize}
    \item Parametry služby a způsob vyhodnocení -- důležitá preciznost jejich vymezení.
    \begin{itemize}[noitemsep]
        \item Z pohledu parametrů je důležité přesně vymezit, kdy se využijí (např.\ výjimka z dostupnosti pro plánované odstávky).
        \begin{itemize}[noitemsep]
            \item Dostupnost
            \item Reakční doba
            \item Doba do odstranění závad
        \end{itemize}
        \item  Z pohledu vyhodnocení je důležité jak bude provedeno vyhodnocení kvality služby
        \begin{itemize}[noitemsep]
            \item Období
            \begin{itemize}[noitemsep]
                \item Rok
                \item Měsíc
                \item Týden
                \item Apod.
            \end{itemize}
            \item Jaký mechanismus
            \begin{itemize}[noitemsep]
                \item Jak bude měřena dostupnost služby
            \end{itemize}
        \end{itemize}
    \end{itemize}
    \item Kreditace -- klíčový prvek, forma sankce za nedodržení úrovně služby
    \begin{itemize}[noitemsep]
        \item Podoba
        \begin{itemize}[noitemsep]
            \item  Sleva z ceny
            \item smluvní pokuta -- může překročit smluvní částku za službu na rozdíl od slevy
            \item případně délku budoucího období poskytnutí služby zdarma tzv free service days
        \end{itemize}
    \end{itemize}
\end{itemize}


\newpage
\section{Jaký je rozdíl mezi zárukou a odpovědností za vady a jak se tyto typicky uplatňují u software?}

\textbf{Záruka} je dobrovolné prohlášení prodávajícího ohledně jakosti jím prodávaného zboží. Při poskytnutí se prodávající zavazuje, že věc bude funkční pro obvyklé použití nebo si zachová její vlastnosti. Stačí pokud je uvedeno na obalu, v reklamě, letáku. Záruku lze poskytnout i jen na část věci.

\textbf{Odpovědnost} dopadá na prodávajícího ze zákona. Její trvání je po dobu dvou let. Kupující uplatňuje právo z vady. Je-li na obalu, návodu, reklamě uvedena doba, po kterou lze věc použít, používá se ustanovení o záruce za jakost. Výše zmíněné neplatí v případě pokud se prodává daná věc už s určitou vadou a její cena je snížena, na opotřebení běžným způsobem, u použité věci odpovídající míře používaní nebo opotřebení, kterou věc měla při převzetí nebo nevyplývá-li to z povahy věci.

U SW se zodpovídá za vady, tyto vady dělíme na faktické a právní. Faktické vady jsou nevhodné či nedostatečné funkcionality SW\@. Mohou souviset s chybami, nevhodnou implementací, nekompatibilitou, bezpečnostními chybami, či nedostatečnou bezpečností SW\@. Právní vady spočívají v zatížení SW nárokem jiné osoby v rozporu se smlouvou, na základě byl SW pořízen. 

Obecně se na SW vztahuje režim odpovědnosti za vady. Z toho vyplývá, že by dodávaný SW měl být bez chyb. Do chyb se počítá odchýlení se od sjednaných vlastností a nevhodnost k výslovně stanovenému účelu. Na bezúplatný SW (Free Open Source Software) se odpovědnost za vady nevztahuje.

Lze domáhat opravy chyby, pokud je to možné, nebo přiměřené slevy z ceny. Pokud není možné vadu odstranit lze odstoupit od smlouvy nebo požadovat snížení ceny.


\section{Co je podstatou softwarových veřejných licencí a jak tyto fungují po právní stránce? Kdy zvolíte jakou veřejnou softwarovou licenci a proč?}

Veřejná licence je specifickým způsobem sjednaná licenční smlouva. SW licencovaný pod veřejnou licencí je vetšinou poskytován bez úplaty, tímto způsobem se lze zbavit odpovědnosti za chyby v programu, které nezpůsobují právní vady. Obsahuje podmínku uvedení autora. 

Podstatou veřejné licence je zveřejnění díla s licenčními podmínkami, odkazem na ně. Kde nabyvatel licence není v přímém kontaktu s poskytovatelem. A využívá se hlavně v situaci kdy licenci chceme směřovat na neurčitý počet osob. \uv{\emph{Veřejné licence jsou veřejné návrhy k uzavření licenčních smluv, jejichž obsah je standardizován a vymezen odkazem na veřejně známé a dostupné licenční podmínky a určen neurčitému počtu osob}}.

Nejčastěji se veřejných licencí využívá ve FOSS\@. Typy licencí mohou být silně copyleftové, slabě copyleftové a necopyleftové.

\textbf{Silně copyleftové} nesou omezené při zpracovaní a šíření SW\@. Požadují, aby původní, nebo nový, program, který obsahuje původní, byl šířen pod původními licenčními podmínkami a současně garantují tvůrci přístup ke zdrojovému kódu. Zástupci jsou GNU GPL v2 a v3.

\textbf{Slabě copyleftové} vyžadují šíření odvozených programů pod stejnými licenčními podmínkami a zpřístupnění jejich zdrojových kódů. Umožňují vytváření programů, které jsou propojené a šířené společně s původním programem aniž by měnily či používaly jeho zdrojový kód a tyto programy šířit pod libovolnou licencí. Nejčastěji to jsou standardní knihovny. Nemusí se vydat zdrojové kódy vlastního kódu ale pouze musí uvést a zpřístupnit původní část programu pod původní licencí. Zástupci MPL (Mozila Public License) v 1.1 a  LGPL (Lesser General Public License) v2.1.

\textbf{Necopyleftové} licence neobsahují žádnou nebo velmi omezenou copyleftovou doložku. Ukládají pouze minimální omezení k dalšímu šíření. Proto lze použít i při vývoji SW s neveřejným zdrojovým kódem aniž by bylo porušeno původních podmínek. Zástupci Apache 2.0, BSD a MIT\@.

Licence lze měnit směrem od nejslabší po nejsilnější ale ne naopak. Další často používanou licencí je Creative Commons.


\section{Definujte správce osobních údajů a popište jeho základní povinnosti dle GDPR\@. Jaký rozdíl mezi správcem a zpracovatelem osobních údajů?}

Správce po většinou chce sbírat osobní údaje a sbírá je za předem definovaným účelem. Lze mít více správců na jedny data aka každý odpovídá sám za sebe.

Zpracovatel je osoba/firma, která je najata správcem osobních údajů. Zpracovatel nemusí vždy existovat nebo jich může být více. Například je možné, že se osobní údaje nezpracovávají, nebo si je správce zpracovává sám. Pokud zpracovatel začne rozhodovat o účelu dat sám, stává se správcem. Mezi těmito entitami musí při zpracovaní být vždy sepsána písemná smlouva.

Příklad -- Máme firmu A, která prodává zboží a bere si osobní údaje jako např.\ věk. Tím se firma A se stává správcem osobních údajů. Firma A chce zpracovat, jakého věku jsou zákazníci nejčastěji nakupující určité položky. Firma A zadá zpracování firmě B nebo osobě, aby jí data zpracovala. Firma B se tím pádem stává zpracovatelem těchto osobních údajů.
\newline

\noindent\textbf{Základní povinnosti správce osobních údajů vyplývající z GDPR jsou:}
\begin{itemize}[noitemsep]
    \item Odpovědnost
    \begin{itemize}[noitemsep]
        \item Za dodržování zásad zpracování
        \item Za dodržování povinností upravených nařízením
        \item Za zabezpečení údajů
    \end{itemize}
    \item Povinosti
    \begin{itemize}[noitemsep]
        \item Aplikace standardní ochrany osobních údajů
        \item Jmenovat pověřence pro ochranu osobních údajů
        \item Posuzovat vliv na ochranu osobních údajů
        \item Ohlásit případy porušení zabezpečení osobních údajů příslušnému úřadu a postiženým osobám
        \item Vést záznamy
    \end{itemize}
\end{itemize}

Ke zpracování osobních údajů je nutno uvést souhlas, kde by mělo být uvedeno, co je shromažďováno a za jakým účelem. Při potřebě může osoba, o které jsou shromážďovány osobní údaje, požádat o vymazání z databáze nebo jen přístup k datům o ní vedeným. Jsou i výjimky, kdy lze zpracovávat osobní údaje bez souhlasu, musí k nim ale existovat zákonný důvod.

\newpage 
\noindent\textbf{Příklad výjimek:}
\begin{itemize}[noitemsep]
    \item Plnění smlouvy
    \item Plnění právní povinnosti -- uchování faktury
    \item Při výkonu veřejné moci
    \item Ochrana životně důležitých zájmů subjektu údajů nebo jiné FO -- lékař uschovává informace o léčbě
    \item Plnění úkolu prováděného ve veřejném zájmu
    \item Nezbytné pro účely oprávněných zájmů příslušného správce -- například při půjčení peněz někomu 
\end{itemize}


\section{Jakými prostředky a čeho se může domáhat autor software při zásahu do jeho autorských práv?}

Ochrana majetkových hodnot je primárně pomocí soukromněprávního vymáhání. Mělo by být dosaženo vrácení předchozí stavu, jako by se nikdy nic nestalo.
\newline

\noindent\textbf{Vymáhat lze pomocí:}
\begin{itemize}[noitemsep]
    \item Soukromněprávního vymáhaní -- napravení předešlého stavu, obsahuje i případ že by zde docházelo ke kompenzacím (inclusive -- použito v přednášce), náhradě škod, vydání
    \item Veřejněprávního vymáhaní
    \begin{itemize}[noitemsep]
        \item Správněprávní - zde chceme člověka potrestat 
        \item Trestněprávní - zde chceme potrestat a domáhat se náhrady škody - často předáno do občanskoprávního řízení
        \item Ústavněprávní
    \end{itemize}
\end{itemize}

\noindent Obecně veřejněprávní chce potrestat osoby dle dohodnutých ustanovení (zákony).

Soukromněprávní řízení je dle AZ nebo průmyslová práva (patenty). Žaloba se podává na krajský soud v místě bydliště žalovaného. Před zahájením sporu by se měly zjistit informace a podat předžalobní výzvu. Poté podáváme žalobu. 

Žaloba by neměla mít nejasný rozsah. Mělo by být udáno, čeho chceme dosáhnout v tzv.\ petitu. Lze se domoci předběžných opatření, které slouží aby nevznikalo další porušovaní práv v průběhu řízení. Vymáhání dle AZ je ohrožovací delikt takže k zásahu ještě nemuselo dojít. 
\newline

\noindent\textbf{Příklady nároků:}
\begin{itemize}[noitemsep]
    \item Určení svého autorství
    \item Sdělení údajů (např.\ o dotyčných osobách, ceně služby)
    \item Odstranění následků zásahu do práva
    \item Poskytnutí zadostiučinění
    \item Uveřejnění rozsudku (např.\ veřejná omluva)
    \item Náhrada škody + kompenzace bezdůvodného obohacení
\end{itemize}

Specifickým nárokem je uveřejnění omluvy. V rámci kompenzace se řeší újma majetková/nemajetková. Do nemajetkové újmy spadá zásah do nemorálních osobnostních práv, v oblasti majetkové újmy je řešeno satisfakcemi možno řešit i finančně. Majetková škoda se dělí na scházející a ušlý zisk -- řeší se to podle občanského ne autorského zákona.

Správněprávní rovina -- řeší se pomocí úřadu obce s rozšířenou působností -- přestupky. Přestupek vs trestný čin -- mírnější trestu a je možné trestat větší okruh jednaní, existuje v něm nedbalost. Trestný čin je spáchán pouze úmyslně. 

Ochrana spotřebitele -- neřeší se kompenzace, vetšinou se jedná pouze o tresty. Padělek -- ochranné známky, nedovolená napodobenina -- autorské právo a průmyslové vzory. SW vypálen na černo spadá do nedovolené napodobeniny.

Trestněprávní rovina -- mělo by to být poslední řešení ne přeskakovat před nižšíma. Řešit důležité věci. U autorského práva není potřeba minimální škoda a řeší ho §270. 


%Odpovědi a otázky jsou použity pro studíjní účely\
%Zdroje:
%+ Právo informačních technologií ISBN 978-80-7598-045-8
%+ BP [Limity autorskoprávní ochrany počítačových programů](https://is.muni.cz/th/p7o93/DP_final.pdf)
%+ dále zápisky z přednášek BPC-SPR
